\chapter{Design}\label{design}

The design of the application will draw from the above requirements and
the associated use cases to form a final application. There are many
decisions to be made when designing an application, each decision needs to be solved in the best way possible given the conditions and the available solutions and development time.

In this chapter some design problems are listed and a discussion is provided for each on a selection of solutions. Each candidate solution is evaluated and compared to other solutions, and a summary of the chosen approach is given in conclusion.

\section{Structure of the information in the solution}\label{structure-of-the-information-in-the-solution}

The application will store notes and code (files) created by the user, however
there are many different ways to store these files. Here are a few of the
different ways to store this information with some positive and negatives of each approach.

\subsection{List of files}\label{list-of-files}

The application could store the notes as a list of simple files in the system.
Each file would have a name and a collection of other attributes: date created,
date modified. This method is the most simple of solutions discussed here however, it does rely on the user to correctly name the files with discipline. If this is done correctly then the list of files solution could be viable.

For example, the following is an idea of how this might look.

\begin{verbatim}
    - Project A - Overview - Project summary
    - Project A - Overview - Change log
    - Project A - Users - User login module
    - Project A - Users - User login module version 2
    - Project B - Overview - Project summary
    - ...
\end{verbatim}
Each line in the above example is a file and each has been named with a logical structure of $$ PROJECT - CATEGORY - NAME $$

For many reasons this is not an ideal way to view files for a single section of a system. Typically developers work with a single project at once and within that project only a limited subset. This means when the developer has to navigate to a particular file in the system they have to find just one file in what can be a large list.


As a positive consequence of choosing a simple list, the times in which need to
access files across multiple projects. For example in a use case where the developer
needs to switch between projects frequently, the developer can see the
entire list of files without the need to change between folders.

\subsection{Folder structure}\label{folder-structure-i.e.explorer-finder}

There are already existing methods for storing files into folders or
directories. Most mainstream operating systems use a hierarchy of
folders\footnote{Windows has Explorer, and OSX has Finder} to contain files
and more folders. This acts like a tree, each leaf is a file and each subtree
is a folder.

This approach still offers the users free reign over the organization
and naming of each file in the system, however, it also provides the
ability to focus on single parts of the tree at once.

\begin{verbatim}
    - Project A/
      - Overview/
        - Project summary
        - Change log
      - Users/
        - User login module
        - User login module version 2
    - Project B/
      - Overview/
        - Project summary
\end{verbatim}

In the example above the items ending with / are the folders (subtrees) and the
other items are the files (leaf nodes).

This organization into folders can take time to complete and can be done in many
different ways. Within each project, unless managed correctly, there could exist
different ways of creating / naming the folders. This problem of uniformity
crops up in many places within the topic of organization and in general, the
more uniform the structure, the less the user has to remember for each case and
less effort to understand it.

There is another problem with a simple tree folder structure, each file in the
folder tree can only exist in one place at once.%
%
\footnote{This can however be solved, as with the unix file system for example.
Symbolic links can allow for files to exist in multiple places at once. } Some
files could belong in multiple places at once. This causes ambiguity and can
make finding these files take longer. Each ambiguously located file that needs
to be found needs to be searched for in (worst case) all the locations it could
be in.

Both the two above problems are lessened by putting sensible limits on
the depth of the folder structure. The reason for this is discussed
below.

\subsection{Simplified folder structure}\label{simplified-folder-structure}

In a simplified folder structure, there are a limited number of levels that can
exist. If viewed as a tree it would have limited depth and contains leaf nodes
(here notes) at the bottom level of the tree.

This simplified structure can help make the choice of folders at each
level more apparent for the user. They have limited choice of
where to place folders and files and as such, the ambiguity problems of a
full folder structure are lessened.

This technique also implicitly limits the user to a set number of
levels. This, although limiting, reduces the number of cases where files have
 more than one obvious location.

\subsection{Tagged files}\label{tagged-files}

In a tagged file organizational system, files are not placed in folders
but instead have each an associated set of ``tags''. Each of these tags
has a name and can be applied to many files.

For example, you might have the following set of files and their tags:

\begin{verbatim}
    - Project summary {Project A, overview}
    - User login module {Project A, Users}
    - Overview {Project A, overview}
    - Project summary {Project B, overview}
\end{verbatim}

The user can then find subsets of files they are interested in by
filtering for the items that have the tags specified:

\begin{verbatim}
    Filter: file has all {Project A, overview}
    Result:
    - Project summary {Project A, overview}
    - Overview {Project A, overview}

    Filter: file has overview
    Result:
    - Project summary {Project A, overview}
    - Overview {Project A, overview}
    - Project summary {Project B, overview}
\end{verbatim}

nb. the folder structure as discussed above
\ref{folder-structure-i.e.explorer-finder} can be represented with a tagged list
of files. This can be achieved by adding a single tag to each file with the path
of the folder that contains it. However the tagging system allows for more
flexibility by allowing multiple tags to exist on the same file.

For some applications, a tagged file system can be both faster and more simple
to understand for the users, the paper by Bergman, Ofer et. al.\cite{ASI:ASI22906}
has a detailed analysis of this technique.

\subsection{Choice of file organization}\label{choice-of-file-organization}

The simplified folder structure was chosen because of the balance
between the flexible organization of the tagging system and the
simplistic file list. The simplified folder structure proved to be
simple to implement and provides the required functionality.


\begin{figure}
\begin{subfigure}{.5\textwidth}
  \centering
  \includegraphics[width=.8\linewidth]{Figures/BinderEmpty.png}
  \caption{}
  \label{fig:binderempty}
\end{subfigure}%
\begin{subfigure}{.5\textwidth}
  \centering
  \includegraphics[width=.8\linewidth]{Figures/BinderFull.png}
  \caption{}
  \label{fig:binderfull}
\end{subfigure}
\caption{Binder, top level container for tabs. fig \ref{fig:binderempty} and \ref{fig:binderfull} show empty and full binders respectively}
\label{fig:binder}
\end{figure}

\begin{figure}
\begin{subfigure}{.5\textwidth}
  \centering
  \includegraphics[width=.8\linewidth]{Figures/PagesEmpty.png}
  \caption{}
  \label{fig:pagesempty}
\end{subfigure}%
\begin{subfigure}{.5\textwidth}
  \centering
  \includegraphics[width=.8\linewidth]{Figures/PagesFull.png}
  \caption{}
  \label{fig:pagesfull}
\end{subfigure}
\caption{Tabs, the second level container. fig \ref{fig:pagesempty} and
\ref{fig:pagesfull} show an empty and full tab respectively. This is shown
when a tab is selected, the pages are listed with the ability to edit the list,
and add new items.}
\label{fig:pages}
\end{figure}

Fig \ref{fig:binder} shows the implemented simple folder structure. The names
given to the levels are as follows: Binder, top level container. Tab, second
level container. Fig \ref{fig:pages} shows how the second level container is
displayed.

\section{Search capability in the system}\label{search-capability-in-the-system}

The requirements gathering process clearly identified in multiple places the
need for the system to provide a quick and simple search function.

There are multiple ways to provide search functionality within a document
management system, some of which are suitable for the application and some are
not. The below is a summary of a few of the options and an evaluation of each.

The search feature should take the users query and provide access to the
matching files in the system. This part of the searching experience will be the
same for all the discussed methods. However the parts that differ in the
compared methods are:
\begin{itemize}
  \item the format of the users query
  \item the searching method
  \item the time taken to perform the search
  \item the presentation of the results
\end{itemize}

\subsection{Simple text search / serial scanning}%
\label{simple-text-search-serial-scanning}

The simplest method for searching in a collection of documents is taking a
simple query from the user (the search term), and searches for the occurrences
of the term in each file. Each match results in an item in the result list
displaying the name of the file, and the position within the document.

For example:

\begin{alltt}
    search term: \emph{test}
    results:
    - File A, line 13, chars 34-38
      - ... the users will then report on \emph{test}ing ...
    - File A, line 13, chars 10-13
      - Chapter 2: \emph{Test}ing ...
\end{alltt}

This technique is simple and for small amounts of files and content is fast to
compute. However, due to the fact that the algorithm is loosely linear in the
number of files this technique would be impractical for large numbers of files.
This technique, called ``serial scanning''\footnote{%
\url{https://en.wikipedia.org/wiki/Full_text_search\#Indexing}%
has more details on this.}%
, is only typically used for smaller amounts of text due to this problem.

\blockquote{ However, when the number of documents to search is potentially
large, or the quantity of search queries to perform is substantial, the problem
of full-text search is often divided into two tasks: indexing and searching.
}\cite{wiki:full-text-search}

This technique that wikipedia refers to is discussed below.

\subsection{Full text index}\label{full-text-index}

When the simple method of searching through all the files becomes too slow, one
alternative is to create a concordance
\footnote{\url{https://en.wikipedia.org/wiki/Concordance_(publishing)} for more
information}. This is a table of words in a publication (here a note) and where
those words appear within the document. By using this concordance in place of
searching the text directly, the application is able to search a smaller amount
of data to find the results and hence speed up the search process.

This requires that there is a pre-processing step before a search can be
completed. This needs to be recalculated each time a user changes a
document so that all subsequent queries can find the new content.

This is the tradeoff with the more complex methods for searching. This method
requires the index to be built / rebuilt before any searching can take place.
However this can be done ``offline'' and cached for later use.

\subsection{Extraction of keywords}\label{extraction-of-keywords}

Another method of searching the documents is by analysis of keywords. Indexing a
select few words per note instead of all the words as in \ref{full-text-index}
means smaller indexes and faster searches.
Either a list of keywords is created and left static or, machine learning
algorithms can learn which words are important within the page and index only
those words.

This is however only applicable when the user's search terms can be limited as
such. This is because if the user searches for a word that is contained within a
file in the application, it will only be found if the word is in the keywords
list. In the case of a note taking application, a search method that can fail to
find a word that exists in the  document is of little use. However indexes of
the popular search terms can help to find preliminary results for a query, or
provide caches of common queries.


%\subsection{Edit distance}\label{edit-distance}

%\begin{quote}
%JP maybe discuss fuzzy search
%\end{quote}

\subsection{Reasons for selecting simple text
search}\label{reasons-for-selecting-simple-text-search}

The numerous ways to search the text within a collection of documents each
provide their own ways to provide the results. While many of the different
search methods are concerned with the performance of a search and their accuracy
in terms of the results, the most basic search discussed the Simple text search
\ref{simple-text-search-serial-scanning} as discussed can offer the features
needed by most users of the application, While remaining simple to implement.
This could always be replaced or enhanced as a future development of the
system.

\section{Type of user interface}\label{type-of-user-interface}

The different types of user interface that designers can choose from
when designing applications can be categorized into a few main
categories:

\begin{itemize}
  \item Command line
  \item GUI / WIMPS (windows, icons, menus, and pointers)
\end{itemize}

For visualizing text, popular command line applications for editing text like
Vim\footnote{http://www.vim.org/} or GNU
Emacs\footnote{\url{https://www.gnu.org/software/emacs/}} do exist however, the
richer environment of GUI applications are predominately preferred with their
ability to customize fonts and other attributes of the layout and style. They
also provide a better experience for users with the use of menus etc. visual
reminders on how to perform actions. These can provide a more simple to use
application than an alternate command line program with many shortcuts to
memorize.

There are many different types of GUI application that can be built and
an important decision that needs to be made before selecting the layout
of the application is what platform to build the application on.

\subsection{Native solution for operating
system}\label{native-solution-for-operating-system}

The first option is to develop a ``desktop first'' application. This would mean
developing an application for a chosen operating system. This would require
selecting one of the native windowing API's for the operating system. Some of
these API's are cross platform enabled\footnote{for example, see ``qt'' for a
popular choice. \url{http://www.qt.io/}} others are proprietary and more
integrated and tied down to the selected operating system. Apple's own windowing
API Cocoa\footnote{\url{https://en.wikipedia.org/wiki/Cocoa_(API)}} is an
example of this.

One advantage of using the manufacturer built API is in the level of support
from the manufacturer that is available. However a downside of this is often the
proprietary libraries are closed source. These closed source API's impose a
limitation towards changing and tweaking the system.

As one of the requirements of the project is that the system should be
accessible to all platforms \footnote{ see \ref{requirement:accessible} for the
requirement} the proprietary method is not an option for a time constrained
development like this. In order to complete a cross platform version using
operating system specific (non cross platform) API's, Multiple versions would
have to be created, one for each different API.

\subsection{Mobile solution for Android / IOS}%
\label{mobile-solution-for-android-ios}

The production of a mobile application for either the Android or IOS operating
systems would either require two versions of the application, or the use of
tools like \href{https://www.xamarin.com/platform}{Xamarin}.

Xamerin's goals are as follows:
\begin{verbatim}
"Deliver native Android, iOS, and Windows apps, using existing skills,
teams, and code."
\end{verbatim}\cite{Xamerin}

Tools like this allow for applications to be built for a variety of mobile
platforms with a single code base. In the case of Xamerin, in C\#. The ability
to ``write once run anywhere'' is a big selling point for these tools. However
it would still restrict users into using mobile only to access the application.

\subsection{Browser based solution}\label{browser-based-solution}

With modern advances in browser technology it has become more feasible to create
full desktop replacement applications as website-applications. These websites
are truly cross platform, they can be viewed on any platform with access to a
web browser. Desktop OS or mobile OS's have access to many of the most
recent browsers.

\begin{itemize}
  \item Chrome (Windows, Mac OSX, IOS, Android)
  \item Safari (Mac OSX, IOS, Windows)
  \item Opera (Windows, Mac OSX, IOS \{opera-mini\}, Android \{opera-mini\})
  \item Firefox (Windows, Mac OSX, IOS, Android)
\end{itemize}

There are many different libraries to help develop applications with javascript.
However it is still possible to create more traditional client-server
applications and have a simple lightweight javascript free front end. These
applications tend to be slower in terms of latency between user action and
application response because of there need to communicate every action with a
server over http.

An alternative to the more traditional client-server architecture is the
now popular ``single page javascript application'' or \emph{SPA}. This merges the
lines between the user perceived differences of the desktop and browser
experience.

An SPA is a single web page that functions like a normal native desktop
application. The application have all the features of a desktop application,
they can make use of windows, buttons, and animations etc.

External data is gathered in background http calls (called Ajax
requests) and does not necessitate a full page reload. The absence of
these full reloads and the flash of white as the page loads give the
user a much better experience.

For more information about the use of ajax replacing client server applications
refer to the paper ``Single Page Web Applications'' By Michael S. Mikowski and Josh C.
Powell\cite{garrett2005ajax}.

\subsection{Reason for selecting browser}\label{reason-for-selecting-browser}

For the application being built the choice of platform is between a
cross platform library for a native desktop application and a web based
solution. For reasons of experience with technology and programming
languages developing for the browser was selected as the platform.

The browser application will be accessible from any platform and device
and provides an easy way to update the application in the future.

There are many different libraries that can help when building a javascript
application in the browser. The libraries have some things in common and other
unique points that set them apart from each other. Most of the libraries cover
the standard principles of MVC ``Model View Controller'' or some similar
variant. \footnote{
\href{https://en.wikipedia.org/wiki/List_of_JavaScript_libraries\#Web-application_related_.28MVC.2C_MVVM.29}%
{https://en.wikipedia.org/wiki/List\_of\_JavaScript\_libraries}
has a community populated list of libraries.}

\subsection{Javascript MVC library React.js}\label{javascript-mvc-library-react.js}

In this section, the choice of javascript library React.js will be explained
along with a brief description of how it works in order to help understanding.

React.js is a fairly new open source library from facebook.

\begin{verbatim}
A JAVASCRIPT LIBRARY FOR BUILDING USER INTERFACES
\end{verbatim}\cite{reactjs}

React.js brings the notions of pure functions to the problem of creating and
updating views. In React.js the headache of updating the view when something
happens in the application is handled automatically. This is accomplished by
having the view hierarchy be a result of a pure function of the application's
current state. This has the important consequence of removing the need for the
developer to detail how an action should update the UI. Instead, the developer
needs only to to update the state and React.js will work out what changes are
required by calculating the difference in the old and new view hierarchies.

\subsection{Future React Native}\label{future-react-native}

React.js has a new related project created by the same team called ``React
Native''\footnote{\url{https://facebook.github.io/react-native/} for more
information}. React Native takes the principles of React.js and allows the same
React.js code, with slight tweaks, to run as a native application on both IOS
and Android.

This would provide a simple way to get the application onto mobile operating systems in a
native environment. With a native environment the users would be able to access the application
as they would any other app on the device and not through the browser.

\section{Choice of interface - Notebook
metaphor}\label{choice-of-interface---notebook-metaphor}

The interface of any application need to be simple and easy to
understand. the more complicated the interface the more brainpower needs
to be dedicated to using it. The most successful user interfaces provide
simple intuitive ways for uses to do the actions they require.

\blockquote{%
  Metaphors are the fundamental concepts, terms, and images by which information
  is easily recognized, understood, and remembered.%
}

Metaphor Design for User Interfaces, Marcus, Aaron\cite{Marcus:1998:MDU:286498.286577}

The metaphor chosen for the application is the notebook. A physical
notebook has pages, tabs etc. These are easily understood by
anyone who knows what a physical notebook is. We can extend the notion of a
notebook with concepts from the wider world of books with indexes and
contents pages.

The application will take from this notion of the ``notebook metaphor'' and from
there the application interface will be designed.

\todo{notebook image mapped to uml diagram of application}

\section{Detachment of the SQL interface through custom web
API}\label{detachment-of-the-sql-interface-through-custom-web-api}

Selecting the browser poses some specific problems for a SQL IDE. A SQL IDE
needs to have access to the SQL server in order to execute queries and retrieve
results. There is no support in any of the mainstream browsers for direct
integration with a SQL server database, although they do have limited support
for in browser databases\footnote{\url{http://caniuse.com/\#search=sql-storage}
shows a report for the support of in browser sql in common browsers}.

This therefore requires the creation of some middleware to connect the database
to the application. The commonly used method for transferring data in the
browser is via a JSON web API. There are many methods for connecting to a SQL
server database however, the Microsoft documentation contains C\#, VB, C++
documentation and libraries for querying the database.\cite{sqlserverconnect} I
personally have experience using the C\# connection methods and with this in mind
the middleware was chosen to be a C\# website providing a mapping from JSON to
the database.

\todo{diagram of connection}
