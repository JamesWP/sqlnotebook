\chapter{Conclusion}\label{conclusion}

The application meets all the mandatory goals (11) set out in the use cases.
Out of the two optional requirements, only one was met by the application.
This is due to the lack of remaining development time. The
delays in the development of the features of the application is
partially due to lack of experience with the workflow and the associated
technologies.

The build process especially was time consuming to pick up. This is
partially due to the many options available. The lack of up to date
information is only apparent after a new build step is required. These
delays and set backs are those that only affect the momentum of the
development at the beginning of the project and after the changes have
been implemented development speeds up.

The implementation of the application, after a change of method for
storing the application data, was extensible and offers a good platform
for new features to be developed on. The code is of good quality with
good code separation. The separate modules have an average length of only 80
lines, this short average file size is a good indicator of good quality code

The users who have seen the application have also offered thoughts on
other uses of the application, most common is the view that the
application could be used as an education tool for new programmers to
teach the code and the concepts behind database design.

The designing of the final solution was mostly influenced by personal
issues identified from personal experience working on database projects.
A more structured exploration into issues and solutions with other
developers could have come up with some more diverse ideas. These same
users could have been consulted at various stages of the development and
provided feedback.

Overall the project was a success and the final application has exceeded
initial expectations of what could be accomplished in a third year project.
