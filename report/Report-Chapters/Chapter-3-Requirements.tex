\chapter{Requirements}\label{requirements}

The existing solutions have provided a basis for identifying the
potential problems the system must overcome. Along with other sources
and other base requirements a set of more formal requirements are listed
below.
\section{Requirements}

The requirements will be detailed below, along with a brief reason for the importance.

\begin{requirement}
The application must store notes about the project in various
separate ``files'' within the system. The files can take on any form however it is required that they are separate entities.

The files must exist in separate items so that each can have a well defined topic. This will aid the user in placing notes in the correct locations.
\end{requirement}

\begin{requirement}
The ``files'' stored within the system must be able to store code and also
notes in a format easily read.

For example these files should have the ability to contain syntax highlighted code and also headings lists.
\end{requirement}

\begin{requirement}
The application should be easily accessible from multiple operating systems.

The targeted operating systems should include Windows, Popular flavors of linux, and Mac OSX.
\end{requirement}

\begin{requirement}
The application should have the ability to view the results of queries executed
on the system in table form.

This is the de-facto representation of the results from a SQL query, although it may be possible to view the results as plain text, the requirement is to display them in tabular form.
\end{requirement}

\begin{requirement}{(Optional)}
The ability to detect the type of result and view it in a
more suitable form. The more suitable form could be anything from a
graph to folder structure to a specific viewer for xml documents.

This could be implemented in a way that automatically selects the best result format for the results. However, this could also be completed with a manual selection of the result type that changes the display format.
\end{requirement}

\begin{requirement}
The application should have a mechanism for organizing related files in
a hierarchy. Files that are about a part of the system should be stored
in the same place in the system.

This requirement is non specific so as to not restrict the possible implementations. The need to organize files is an old requirement of many existing systems both specifically related to code and in general. The number of different methods for providing organization is vast.
\end{requirement}

\begin{requirement}
The application should save the previous versions of the files stored within. These
versions should allow editing and saving over newer versions if needed.
\end{requirement}

\begin{requirement}
The application should allow for multiple files to
be open at once and viewed at the same time. This can be done in a free
sized window mode or windows listed vertically or horizontally.
\end{requirement}

\begin{requirement}
The application should allow for searching for strings in
files.
\end{requirement}

\begin{requirement}{(Optional)}
The ability to view in each of the search results, where the
matched text is. And the ability to open the file from the search
results.
\end{requirement}

\begin{requirement}
The application should allow for the easy
comparison of the results of the queries. Historic results of the
queries should be shown as a diff or side by side with newly executed
\end{requirement}

\section{Use cases of the system}\label{use-cases-of-the-system}

The use cases of the system will detail how the system is to be used by
its users. The primary users of the system as discussed is the
developers of the projects themselves. This restriction of the users of
the system and their interaction with it makes the system more simple to
design as there are less interaction to model in the system.

\subsection{Use case 1}\label{use-case-1}

Developer wants to document information about a development.

\subsection{Use case 2}\label{use-case-2}

Developer needs to view all documents regarding a project stored in the
system.

\subsection{Use case 3}\label{use-case-3}

Developer needs to find reference to a particular item within the system
but is unfamiliar with the specific location or other detail

\subsection{Use case 4}\label{use-case-4}

Developer has to find specific details of a system previously worked on
in order to maintain it.

\subsection{Use case 5}\label{use-case-5}

Developer wants to view and compare previous versions of some code
within the database

\subsection{Use case 6}\label{use-case-6}

Developer want to view the history of the notebook. There is a
particular entity of the system e.g.~table that they want to view the
history. i.e.~how the entity has evolved in the past.

\subsection{Use case 7}\label{use-case-7}

The developer wants to find the places a particular entity is used
within the system to better understand how the changes planned could
effect the system as a whole. AKA (impact
analysis){[}https://en.wikipedia.org/wiki/Change\_impact\_analysis{]}. *
cite impact analysis* \#\# Development priority of requirements
Discussion of importance of requirements identified above, put into
order and possibly identify partial fulfillment ideas
